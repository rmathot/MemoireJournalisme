% mainfile: document.tex

\chapter{Le journalisme de données en Belgique francophone}

\section{Les petites expériences de \textit{L'Avenir}}
La seule tentative de journalisme de données en Belgique francophone qui soit documentée dans la littérature est celle menée au sein du groupe \textit{L'Avenir} \cite{handbookfr}.

La (très petite) rédaction Huy-Waremme de \textit{L'Avenir} y pratique le journalisme de données, à petite échelle, depuis 2012. Les journalistes qui voulaient réaliser l'un ou l'autre projet de journalisme de données devaient continuer à produire leurs articles habituels pour le journal papier et le site Internet.

L'accent est donc principalement mis sur de petites visualisations, réalisés à partir d'outils existants et en moins d'une journée. Deux exemples parmi d'autres: un graphique présente la répartition des subsides aux clubs sportifs locaux\footnote{\url{http://www.lavenir.net/article/detail.aspx?articleid=DMF20121211_00243105}} et une ligne du temps revient sur les déboires judiciaires d'Anne-Marie Lizin\footnote{\url{http://www.lavenir.net/article/detail.aspx?articleid=DMF20120426_026}}.

Partant de ces expériences de taille réduite, la rédaction a préparé plusieurs documents interactifs\footnote{\url{http://www.lavenir.net/extra/dommages-de-guerre}} sur la Première Guerre mondiale, avec l'aide de ses lecteurs pour la collecte des informations.

Au niveau régional, le site Internet \url{http://www.lavenir.net} propose notamment une base de données des communes de Wallonie et de Bruxelles\footnote{\url{http://www.lavenir.net/extra/categories-menu-communes}}. On peut y retrouver différentes données chiffrées à propos de chaque commune, dans les domaines économiques, démographiques, politiques\dots comme le revenu moyen par habitant, la composition du collège communal ou le nombre d'habitants. Ici, les données proviennent de sources officielles et la visualisation des données est réalisée principalement au moyen de cartes, dont les couleurs varient pour indiquer les disparités entre les différentes municipalités. 

\section{\og Offshore Leaks \fg, l'enquête en réseau mondial}

L'affaire \og Offshore Leaks \fg a éclaté au début avril 2013 avec la publication d'un grand nombres d'informations concernant des comptes situés dans les paradis fiscaux. 

Elle découle du travail de différents journalistes mondiaux, qui ont travaillé en réseau par le biais de l'\textit{International Consortium of Investigation Journalists} (ICIJ). En Belgique, c'est un journaliste du \em Soir\em, Alain Lallemand, qui a réalisé le travail concernant les potentiels personnalités ou sociétés belges qui ont cédé à l'évasion fiscale.

Selon les déclarations des représentants de l'ICIJ \cite{OL}, ils ont reçu anonymement un disque dur, contenant de nombreux courriels, contrats, listings\dots renseignant les clients de différentes banques de paradis fiscaux, pendant des dizaines d'années. La phase d'acquisition des données a donc été plutôt simple à réaliser.

Le nettoyage des 2,5 millions de données a, lui, demandé plusieurs mois de travail \cite{OL}, car les fichiers informatiques n'étaient pas dans un format directement exploitable. Les journalistes de l'ICIJ se sont notamment adjoint l'aide de développeurs informatiques, pour réaliser des logiciels à même de trier les données.

La publication des données s'est faite de manière plus classique, par la voie d'articles écrits \footnote{Le site Internet du \textit{Soir} conserve un dossier regroupant tous les articles publiés en Belgique : \url{http://www.lesoir.be/tag/offshore-leaks}.}, publiés de manière concomittantes dans le monde entier. Enfin, l'ICIJ a rendu accessible en ligne\footnote{\url{http://offshoreleaks.icij.org/}} sa base de données, permettant à tout un chacun de l'explorer.

\og Offshore Leaks \fg est un exemple-type de journalisme de données à grande échelle. 
Partant de fichiers informatiques qui lui sont tombés dans les mains, un réseau mondial de journalistes a travaillé plusieurs mois, de concert avec des développeurs informatiques, pour parvenir à extraire des informations significatives, mettant ainsi au jour un gigantesque scandale fiscal.

\section{Les journaux financiers}

La plupart des journaux économiques et financiers fournissent des informations en ligne 
sur les cours de bourse ou les entreprises cotées, consultables de manière interactive par les internautes.

Par exemple, le magazine \textit{Trends-Tendances} propose une base de données\footnote{\url{http://trendstop.levif.be/fr/}} contenant diverses informations sur les entreprises : éléments de bilan, classements, coordonnées, historiques. Ces informations --en partie payantes-- sont principalement à destination des investisseurs. 

Autres exemples: \textit{L'Écho} propose un \og tableau de bord des données macro-économiques belges \fg\footnote{\url{http://www.lecho.be/service/tableaudebord}}, qui permet de visualiser en un coup d'oeil divers indicateurs économiques concernant notre pays, ainsi que leur évolution récente. \textit{L'Écho} propose aussi une carte des prix de l'immobilier interactive\footnote{\url{http://monargent.lecho.be/service/carteimmo}}, sur laquelle le lecteur-internaute peut visualiser les régions les plus avantageuses en termes de prix des terrains ou des villas.

\section{La base de données de mandats de Cumuleo}

Le site internet \url{http://www.cumuleo.be/} --qui n'a pas été créé par un journaliste 
ou une rédaction-- 
recense la liste des mandats détenus par 
les politiciens belges, liste qu'ils sont tenus de publier annuellement. Cumuleo 
récupère les publications au Moniteur belge et les traite, pour les présenter comme une  
base de données consultables. Il fait en outre appel au public pour compléter les fiches 
signalétiques du personnel politique. 

Cette base de données, mise à jour chaque été, est régulièrement reprise dans les médias 
belges, qui établissent divers classements des \og cumulards \fg. Dernier exemple en 
date: \textit{Le Vif/L'Express} a réalisé un classement\footnote{Ettore \textsc{Rizza}, \textit{Le top des vrais cumulards}, 29 août 2013, \\ \url{http://www.levif.be/info/actualite/belgique/le-top-des-vrais-cumulards/article-4000385757429.htm}.} des hommes et femmes politiques,
sur d'autres critères que le nombre de mandats (l'importance relative des mandats et 
leur rémunération notamment).

Cet exemple illustre bien la prise en main des données disponibles par une rédaction, 
qui décide de leur appliquer un traitement original dépassant le simple tri, et présenté de manière visuelle.
