% mainfile: document.tex

\chapter{Le journalisme de données en Belgique francophone}

\section{L'Avenir}
La seule tentative de journalisme de données en Belgique francophone qui soit citée dans la littérature est celle menée au sein du groupe \em L'Avenir\em.

La rédaction Huy-Waremme de \textit{L'Avenir}


Le site Internet \url{http://www.lavenir.net} propose notamment une base de données des communes de Wallonie et de Bruxelles. On peut y retrouver différentes données chiffrées à propos de chaque commune, dans les domaines économiques, démographiques, politiques\dots
Par exemple: le revenu moyen par habitant, la composition du collège communal, le nombre d'habitants.

La visualisation des données est réalisée principalement au moyen de cartes, dont les couleurs varient pour indiquer les disparités entre les différentes municipalités. 

\todo{mettre ici la carte des couleurs des bourgmestres}

nombre de journalistes affectés ?  moyen de collecte des données ?

\section{Offshore Leaks}

L'affaire \og Offshore Leaks \fg a éclaté au printemps 2013\todo{quand}, lorsque différents journalistes dans le monde sont entrés en possession d'informations concernant des comptes situés dans des paradis fiscaux. Ces journalistes ont travaillé en réseau par le biais de l'\em International Consortium of Investigation Journalists\em (ICIJ). En Belgique, c'est \em Le Soir\em qui a réalisé le travail concernant les potentiels évadés fiscaux belges.

Selon les déclarations des représentants de l'ICIJ, ils ont reçu un disque dur, contenant de nombreux mails, contrats, listings... renseignant les clients de différentes banques de paradis fiscaux. La phase d'acquisition des données a donc été plutôt simple à réaliser.

Le nettoyage des données a, lui, demandé xxx mois de travail, car les fichiers informatiques n'étaient pas dans un format directement exploitable. Les journalistes de l'ICIJ se sont notamment adjoint l'aide de développeurs informatiques, pour réaliser des logiciels à même de trier les données.

Au \em Soir\em, c'est le journaliste Alain Lallemand qui a analysé les documents mentionnant des Belges, ou des personnes et des entreprises en lien avec la Belgique.

La publication des données s'est faite de manière plus classique, par la voie d'articles écrits. Les journalistes de l'ICIJ n'ont, dans ce cas, pas publié leurs données brutes. (vie privée)?

\section{Les journaux financiers}

La plupart des journaux économiques et financiers fournissent des informations en ligne 
sur les cours de bourse ou les entreprises cotées, consultables de manière interactive par les internautes.

Par exemple, le magazine \textit{Trends-Tendances} propose une base de données\footnote{\url{http://trendstop.levif.be/fr/}} contenant diverses informations sur les entreprises : éléments de bilan, classements, coordonnées, historiques. Ces informations --en partie payantes-- sont principalement à destination des investisseurs. 

Autres exemples: \textit{L'Écho} propose un \og tableau de bord des données macro-économiques belges \fg\footnote{\url{http://www.lecho.be/service/tableaudebord}}, qui permet de visualiser en un coup d'oeil divers indicateurs économiques concernant notre pays, ainsi que leur évolution récente. \textit{L'Écho} propose aussi une carte des prix de l'immobilier interactive\footnote{\url{http://monargent.lecho.be/service/carteimmo}}, sur laquelle le lecteur-internaute peut visualiser les régions les plus avantageuses en termes de prix des terrains ou des villas.

\section{La base de données de mandats de Cumuleo}

Le site internet \url{http://www.cumuleo.be/} --qui n'a pas été créé par un journaliste 
ou une rédaction-- 
recense la liste des mandats détenus par 
les politiciens belges, liste qu'ils sont tenus de publier annuellement. Cumuleo 
récupère les publications au Moniteur belge et les traite, pour les présenter comme une  
base de données consultables. Il fait en outre appel au public pour compléter les fiches 
signalétiques du personnel politique. 

Cette base de données, mise à jour chaque été, est régulièrement reprise dans les médias 
belges, qui établissent divers classements des \og cumulards \fg. Dernier exemple en 
date: \textit{Le Vif/L'Express} a réalisé un classement\footnote{Ettore \textsc{Rizza}, \textit{Le top des vrais cumulards}, 29 août 2013, \\ \url{http://www.levif.be/info/actualite/belgique/le-top-des-vrais-cumulards/article-4000385757429.htm}.} des hommes et femmes politiques,
sur d'autres critères que le nombre de mandats (l'importance relative des mandats et 
leur rémunération notamment).

Cet exemple illustre bien la prise en main des données disponibles par une rédaction, 
qui décide de leur appliquer un traitement original dépassant le simple tri, et présenté de manière visuelle.
