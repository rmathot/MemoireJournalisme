% mainfile: document.tex

\chapter{Le journalisme de données en Belgique francophone}

\section{L'Avenir.net}
La seule tentative de journalisme de données en Belgique francophone qui soit citée dans la littérature est celle menée par les différents titres du groupe \em L'Avenir\em.

Le site Internet \url{http://www.lavenir.net} propose notamment une base de données des communes de Wallonie et de Bruxelles. On peut y retrouver différentes données chiffrées à propos de chaque commune, dans les domaines économiques, démographiques, politiques\dots
Par exemple: le revenu moyen par habitant, la composition du collège communal, le nombre d'habitants.

La visualisation des données est réalisée principalement au moyen de cartes, dont les couleurs varient pour indiquer les disparités entre les différentes municipalités. 

\todo{mettre ici la carte des couleurs des bourgmestres}

nombre de journalistes affectés ?  moyen de collecte des données ?

\section{Offshore Leaks}

L'affaire \og Offshore Leaks \fg a éclaté au printemps 2013\todo{quand}, lorsque différents journalistes dans le monde sont entrés en possession d'informations concernant des comptes situés dans des paradis fiscaux. Ces journalistes ont travaillé en réseau par le biais de l'\em International Consortium of Investigation Journalists\em (ICIJ). En Belgique, c'est \em Le Soir\em qui a réalisé le travail concernant les potentiels évadés fiscaux belges.

Selon les déclarations des représentants de l'ICIJ, ils ont reçu un disque dur, contenant de nombreux mails, contrats, listings... renseignant les clients de différentes banques de paradis fiscaux. La phase d'acquisition des données a donc été plutôt simple à réaliser.

Le nettoyage des données a, lui, demandé xxx mois de travail, car les fichiers informatiques n'étaient pas dans un format directement exploitable. Les journalistes de l'ICIJ se sont notamment adjoint l'aide de développeurs informatiques, pour réaliser des logiciels à même de trier les données.

Au \em Soir\em, c'est le journaliste Alain Lallemand qui a analysé les documents mentionnant des Belges, ou des personnes et des entreprises en lien avec la Belgique.

La publication des données s'est faite de manière plus classique, par la voie d'articles écrits. Les journalistes de l'ICIJ n'ont, dans ce cas, pas publié leurs données brutes. (vie privée)?

\section{Cas des journaux financiers}

La plupart des journaux économiques et financiers fournissent des informations en ligne sur les cours de bourse ou les entreprises cotées. Dans la plupart des cas, elles sont consultables de manière interactive par les internautes.

Ex Base de données des entreprises de Trends/Tendances.
