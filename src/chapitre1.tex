% mainfile: document.tex

\chapter{Les contours du journalisme de données}


\og Data-driven journalism \fg, \og data journalism(e) \fg, \og journalisme de données
 \fg\dots Ces différents 
termes\footnote{Dans un souci de simplicité et de cohérence pour le lecteur, ce travail
utilise toujours 
le terme \em journalisme de données\em.} 
coexistent pour nommer une nouvelle pratique journalistique, née à la fin des années 2000 \cite{bradshaw}, et les 
méthodes et techniques qui en découlent. 

Le principe de base du journalisme de données est le suivant: faire profiter le travail 
du journaliste de l'abondance de données informatisées disponibles, notamment sur 
Internet. 

Avant d'aller plus avant dans la définition du journalisme de données, il convient tout
d'abord de clarifier ce que l'on entend par \og données \fg.
Le dictionnaire Larousse les définit comme suit : \og Renseignement qui sert de point 
d'appui; Représentation conventionnelle d'une information en vue de son traitement 
informatique.\fg

Dans le même sens, Bradshaw définit les données comme \og des informations qui peuvent 
être analysées par les ordinateurs \fg \cite{bradshaw}. Il s'agit donc pour le 
journaliste de profiter d'une nouvelle source d'informations, desquelles on 
peut extraire un récit expliquant une facette du monde, ou tout au moins un 
point de départ pour une enquête journalistique classique. 


\begin{quote}
\og Les données peuvent être la source du datajournalisme, elles peuvent être l'outil 
qui permet de raconter l'histoire -- ou elles peuvent être les deux. 
Comme n'importe quelle source, elles doivent être traitées avec scepticisme; et comme 
n'importe quel outil, nous devons prendre conscience de leurs limites et de leur 
influence sur la forme des histoires qu'elles nous permettent de créer. \fg  %\,--Paul \textsc{Bradshaw}, dans 
\cite{handbookfr}.
\end{quote}

Ce qui caractérise les données par rapport à d'autres sources d'informations 
journalistiques, c'est leur nombre. Par exemple, \em The Guardian \em s'est 
intéressé aux notes de frais des parlementaires britanniques\footnote{\url{http://www.telegraph.co.uk/news/newstopics/mps-expenses/}} 
et aux potentiels abus qui en 
découlent. Ce sont des centaines de milliers de documents qui ont été épluchés par les 
journalistes du quotidien, mais cette tâche n'aurait probablement pas pu être
réalisée aussi vite sans l'aide d'un ordinateur.

Le journalisme de données essaie donc d'extraire une information signifiante 
d'une masse de données complexes. La collaboration entre journaiste et développeur 
informatique peut vite s'avérer indispensable, même si le journaliste \og bidouilleur \fg 
peut déjà obtenir des résultats intéressants avec certains outils de base comme un 
tableur.

Enfin, la présentation des résultats de l'enquête du journaliste peut bénéficier 
des nouvelles techniques multimédia. Cartes interactives, bases d'informations
que le lecteur-internaute peut lui-même consulter\dots sont de nouvelles
manières pour le journaliste de communiquer son résultat avec le public.

Le journalisme de données semble donc proche du journalisme dans sa conception 
plus classique : il s'agit toujours de rechercher l'information, la vérifier, 
la sélectionner et la hiérarchiser et enfin de la communiquer. 

Le journalisme de données intervient dans ce processus, à l'une ou à plusieurs de ces 
étapes, avec de nouveaux outils de recherche de l'information, de nouvelles 
techniques de vérification et de sélection de celle-ci, et enfin de nouvelles 
formes de narration de l'information.


\section{Historique}

Le journalisme de données semble s'être développé à la fin des années 2000, 
concomittament au mouvement de l'\og open data \fg. 

Le gouvernement belge dispose d'ailleurs de son site "open data" \url{http://data.gov.be}, alimenté par les données des services publics fédéraux et flamands.


Le mouvement "open data"

exemples d'usages

Wikileaks, The Guardian, les cartes de machin chouettes...


Description d'un cas ou l'autre en presse BE: Offshore Leaks, WikiLeaks, L'Echo, L'Avenir

Une réflexion sur les enjeux en terme de vérification et d'accès aux données


Le cycle du data journalisme

Il faut trouver des données, les nettoyer, les comprendre, les visualiser et 
raconter l'histoire
