% mainfile: document.tex

\chapter{Les contours d'un journalisme basé sur les données}


\og Data-driven journalism \fg, \og data journalism(e) \fg, \og journalisme de données
 \fg\dots Ces différents 
termes\footnote{Dans un souci de simplicité et de cohérence pour le lecteur, ce travail
utilise toujours 
le terme \em journalisme de données\em.} 
coexistent pour nommer une nouvelle pratique journalistique, née à la fin des années 2000 \cite{bradshaw}, et les 
méthodes et techniques qui en découlent. 

Le principe de base du journalisme de données est le suivant: faire profiter le travail 
du journaliste de l'abondance de données informatisées disponibles, notamment sur 
Internet. 

Avant d'aller plus avant dans la définition du journalisme de données, il convient tout
d'abord de clarifier ce que l'on entend par \og données \fg.
Le dictionnaire Larousse les définit comme suit : \og Renseignement qui sert de point 
d'appui; Représentation conventionnelle d'une information en vue de son traitement 
informatique.\fg

Dans le même sens, Bradshaw définit les données comme \og des informations qui peuvent 
être analysées par les ordinateurs \fg \cite{bradshaw}. Il s'agit donc pour le 
journaliste de profiter d'une nouvelle source d'informations, desquelles il 
peut extraire un récit expliquant une facette du monde, ou tout au moins un 
point de départ pour une enquête journalistique classique. 


\begin{quote}
\og Les données peuvent être la source du datajournalisme, elles peuvent être l'outil 
qui permet de raconter l'histoire -- ou elles peuvent être les deux. 
Comme n'importe quelle source, elles doivent être traitées avec scepticisme; et comme 
n'importe quel outil, nous devons prendre conscience de leurs limites et de leur 
influence sur la forme des histoires qu'elles nous permettent de créer. \fg  %\,--Paul \textsc{Bradshaw}, dans 
\cite{handbookfr}.
\end{quote}

Ce qui caractérise les données par rapport à d'autres sources d'informations 
journalistiques, c'est leur nombre. Par exemple, \em The Telegraph \em s'est 
intéressé aux notes de frais des parlementaires britanniques\footnote{\url{http://www.telegraph.co.uk/news/newstopics/mps-expenses/}} 
et aux potentiels abus qui en 
découlent. Ce sont des centaines de milliers de documents qui ont été épluchés par les 
journalistes du quotidien, mais cette tâche n'aurait probablement pas pu être
réalisée aussi vite sans l'aide d'un ordinateur.

Le journalisme de données essaie donc d'extraire une information signifiante 
d'une masse de données complexes. La collaboration entre journaliste et développeur 
informatique va vite s'avérer indispensable; même si, au départ, le journaliste \og 
bidouilleur \fg peut déjà obtenir des résultats intéressants avec certains outils de 
base comme un tableur.

En outre, la présentation des résultats de l'enquête du journaliste peut bénéficier 
des nouvelles techniques multimédia. Cartes interactives, bases d'informations
que le lecteur-internaute peut lui-même consulter\dots sont de nouvelles
manières pour le journaliste de communiquer son résultat avec le public. \cite{handbook,bradshaw}

Le journalisme de données semble donc proche du journalisme dans sa conception 
plus classique : il s'agit toujours de rechercher l'information, de la vérifier, 
de la sélectionner et de la hiérarchiser et enfin de la communiquer. 

Le journalisme de données intervient dans ce processus, à l'une ou à plusieurs de ces 
étapes, avec de nouveaux outils de recherche de l'information, de nouvelles 
techniques de vérification et de sélection de celle-ci et enfin de nouvelles 
formes de communication de l'information. \cite{handbook}


\section{Une courte histoire du journalisme de données}

Le journalisme de données est un phénomène récent, qui s'est popularisé à la
fin des années 2000. L'un des pionniers du journalisme de données est Adrian Holovaty. Dès 2006, 
ce journaliste et développeur web américain propose \cite{holovaty} que les sites d'informations publient les informations 
brutes, dans un format facilement traitable par un ordinateur \cite{handbookfr}.
Son exemple: en plus de publier un article traitant d'un incendie dans le 
journal local, autant publier un site recensant tous les incendies du quartier 
et permettant de les trier selon une série de critères par type, lieu, date\dots
Il réalisera ensuite une série de \og cartes du crime \fg de 
Chicago. Ces cartes permettaient\footnote{Le site \em ChicagoCrime\em n'est plus en ligne aujourd'hui.} à chacun vérifier quels crimes étaient commis dans le voisinage, et de se faire une idée de la dangerosité de l'un ou l'autre quartier.


Les racines du journalisme de données remontent pourtant à plus d'un demi-siècle.
Dès les années 1960, les journalistes américains s'intéressent aux techniques de
\em computer-assisted investigative reporting \em (journalisme assisté par ordinateur)
Ils veulent pouvoir s'affranchir des communiqués officiels et vérifier --tant que
faire ce peut-- l'exactitude des chiffres présentés. Dans la même veine, dans les
années 1970, Philip Meyer propose d'adapter les techniques quantitatives des 
sciences sociales au journalisme \cite{meyer}.

Le journalisme de données a profité d'Internet, qui offre la possibilité de
mettre des données à disposition de tous, sous forme informatique. Concomitamment aux 
premières expérimentations de Holovaty et ses confrères, le mouvement de 
l'\em open data\em\,(littéralement : \og données ouvertes \fg) revendique un
accès aux données des administrations publiques. Dans la plupart des régimes 
démocratiques, il est prévu, au moins théoriquement, que les autorités rendent 
des comptes au citoyen, en permettant l'accès aux actes de l'administration
(\em Freedom of Information Act\em\,aux États-Unis, directive européenne \og Informations du secteur public \fg de 2003\dots).
Face aux demandes pour plus de transparence des données, de nombreux pays ont
ouvert des sites qu'ils alimentent progressivement en données diverses et variées.
À titre d'exemple, le gouvernement belge dispose de son site de données ouvertes\footnote{\url{http://data.gov.be/}}. Il est actuellement en test et alimenté par les données des services publics fédéraux et flamands. Un site similaire existe pour l'Union Européenne\footnote{\url{http://open-data.europa.eu/}}



Le mouvement "open data"

exemples d'usages

Wikileaks, The Guardian, les cartes de machin chouettes...


Description d'un cas ou l'autre en presse BE: Offshore Leaks, WikiLeaks, L'Echo, L'Avenir

Une réflexion sur les enjeux en terme de vérification et d'accès aux données


Le cycle du data journalisme

Il faut trouver des données, les nettoyer, les comprendre, les visualiser et 
raconter l'histoire
