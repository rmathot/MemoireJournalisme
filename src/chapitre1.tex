% mainfile: document.tex

\chapter{Les contours du journalisme de données}


\og Data-driven journalism \fg, \og data journalism(e) \fg, \og journalisme de données \fg\dots Ces différents 
termes\footnote{Dans un souci de simplicité et de cohérence pour le lecteur, ce travail utilise toujours 
le terme \em journalisme de données\em.} 
coexistent pour nommer une nouvelle pratique journalistique, née à la fin des années 2000 \cite{bradshaw}, et les 
méthodes et techniques qui en découlent. 

Le principe de base du journalisme de données est le suivant: faire profiter le travail du journaliste 
de l'abondance de données numériques disponibles, notamment sur Internet. 

Avant d'aller plus avant dans la définition du journalisme de données, il convient tout d'abord de clarifier ce que l'on entend par \og données \fg.
Le dictionnaire Larousse les définit comme suit : 
\begin{quote}\og Renseignement qui sert de point d'appui; Représentation conventionnelle d'une information en vue de son traitement informatique.\fg \end{quote}

Dans le même sens, Bradshaw définit les données comme \og des informations qui peuvent être analysées par les ordinateurs \fg \cite{bradshaw}.


\begin{quote}
\og Les données peuvent être la source du datajournalisme, elles peuvent être l'outil qui permet de raconter l'histoire -- ou elles peuvent être les deux. 
Comme n'importe quelle source, elles doivent être traitées avec scepticisme; et comme n'importe quel outil, nous devons prendre conscience de leurs limites et de leur influence sur la forme des histoires qu'elles nous permettent de créer. \fg  %\,--Paul \textsc{Bradshaw}, dans 
\cite{handbookfr}.
\end{quote}








\section{Historique}

Si le développement du journalisme de données
Le DataJournalisme, c'est quoi ? Par qui ? Pour quoi ?



Le mouvement "open data"

exemples d'usages

Wikileaks, The Guardian, les cartes de machin chouettes...


Description d'un cas ou l'autre en presse BE: Offshore Leaks, WikiLeaks, L'Echo, L'Avenir

Une réflexion sur les enjeux en terme de vérification et d'accès aux données


Le cycle du data journalisme

Il faut trouver des données, les nettoyer, les comprendre, les visualiser et 
raconter l'histoire
