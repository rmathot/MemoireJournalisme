% mainfile: document.tex

\chapter{Les contours d'un journalisme basé sur les données}


\og Data-driven journalism \fg, \og data journalism(e) \fg, \og journalisme de données
 \fg\dots Ces différents 
termes\footnote{Dans un souci de simplicité et de cohérence pour le lecteur, ce travail
utilise toujours 
le terme \em journalisme de données\em.} 
coexistent pour nommer une nouvelle pratique journalistique, née à la fin des années 2000 \cite{bradshaw}, et les 
méthodes et techniques qui en découlent. 

Le principe de base du journalisme de données est le suivant: faire profiter le travail 
du journaliste de l'abondance de données informatisées disponibles, notamment sur 
Internet. 

Avant d'aller plus avant dans la définition du journalisme de données, il convient tout
d'abord de clarifier ce que l'on entend par \og données \fg.
Le dictionnaire Larousse les définit comme suit : \og Renseignement qui sert de point 
d'appui; Représentation conventionnelle d'une information en vue de son traitement 
informatique.\fg

Dans le même sens, Bradshaw définit les données comme \og des informations qui peuvent 
être analysées par les ordinateurs \fg \cite{bradshaw}. Il s'agit donc pour le 
journaliste de profiter d'une nouvelle source d'informations, desquelles il 
peut extraire un récit expliquant une facette du monde, ou tout au moins un 
point de départ pour une enquête journalistique classique. 


\begin{quote}
\og Les données peuvent être la source du datajournalisme, elles peuvent être l'outil 
qui permet de raconter l'histoire -- ou elles peuvent être les deux. 
Comme n'importe quelle source, elles doivent être traitées avec scepticisme; et comme 
n'importe quel outil, nous devons prendre conscience de leurs limites et de leur 
influence sur la forme des histoires qu'elles nous permettent de créer. \fg  %\,--Paul \textsc{Bradshaw}, dans 
\cite{handbookfr}.
\end{quote}

Ce qui caractérise les données par rapport à d'autres sources d'informations 
journalistiques, c'est leur nombre. Par exemple, \em The Telegraph \em s'est 
intéressé aux notes de frais des parlementaires britanniques\footnote{\url{http://www.telegraph.co.uk/news/newstopics/mps-expenses/}} 
et aux potentiels abus qui en 
découlent. Ce sont des centaines de milliers de documents qui ont été épluchés par les 
journalistes du quotidien, mais cette tâche n'aurait probablement pas pu être
réalisée aussi vite sans l'aide d'un ordinateur.

Le journalisme de données essaie donc d'extraire une information signifiante 
d'une masse de données complexes. La collaboration entre journaliste et développeur 
informatique va vite s'avérer indispensable; même si, au départ, le journaliste \og 
bidouilleur \fg peut déjà obtenir des résultats intéressants avec certains outils de 
base comme un tableur.

En outre, la présentation des résultats de l'enquête du journaliste peut bénéficier 
des nouvelles techniques multimédia. Cartes interactives, bases d'informations
que le lecteur-internaute peut lui-même consulter\dots sont de nouvelles
manières pour le journaliste de communiquer son résultat avec le public. \cite{handbook,bradshaw}

Le journalisme de données semble donc proche du journalisme dans sa conception 
plus classique : il s'agit toujours de rechercher l'information, de la vérifier, 
de la sélectionner et de la hiérarchiser et enfin de la communiquer. 

Le journalisme de données intervient dans ce processus, à l'une ou à plusieurs de ces 
étapes, avec de nouveaux outils de recherche de l'information, de nouvelles 
techniques de vérification et de sélection de celle-ci et enfin de nouvelles 
formes de communication de l'information. \cite{handbook}


\section{Une courte histoire du journalisme de données}

Le journalisme de données est un phénomène récent, qui s'est popularisé à la
fin des années 2000. L'un des pionniers du journalisme de données est Adrian Holovaty. Dès 2006, 
ce journaliste et développeur web américain propose \cite{holovaty} que les sites d'informations publient les informations 
brutes, dans un format facilement traitable par un ordinateur \cite{handbookfr}.
Son exemple: en plus de publier un article traitant d'un incendie dans le 
journal local, autant publier un site recensant tous les incendies du quartier 
et permettant de les trier selon une série de critères par type, lieu, date\dots
Il réalisera ensuite une série de \og cartes du crime \fg de 
Chicago. Ces cartes permettaient\footnote{Le site \textit{ChicagoCrime} n'est plus en ligne aujourd'hui.} à chacun vérifier quels crimes étaient commis dans le voisinage, et de se faire une idée de la dangerosité de l'un ou l'autre quartier.


Les racines du journalisme de données remontent pourtant à plus d'un demi-siècle.
Dès les années 1960, les journalistes américains s'intéressent aux techniques de
\em computer-assisted investigative reporting \em (journalisme assisté par ordinateur)
Ils veulent pouvoir s'affranchir des communiqués officiels et vérifier --tant que
faire ce peut-- l'exactitude des chiffres présentés. Dans la même veine, dans les
années 1970, Philip Meyer propose d'adapter les techniques quantitatives des 
sciences sociales au journalisme \cite{meyer}.

Le journalisme de données a profité d'Internet, qui offre la possibilité de
mettre des données à disposition de tous, sous forme informatique. Concomitamment aux 
premières expérimentations de Holovaty et ses confrères, le mouvement de 
l'\em open data\em\,(littéralement : \og données ouvertes \fg) revendique un
accès aux données des administrations publiques. Dans la plupart des régimes 
démocratiques, il est prévu, au moins théoriquement, que les autorités rendent 
des comptes au citoyen, en permettant l'accès aux actes de l'administration
(\em Freedom of Information Act\em\,aux États-Unis, directive européenne \og Informations du secteur public \fg de 2003\dots).
Face aux demandes pour plus de transparence des données, de nombreux pays ont
ouvert des sites qu'ils alimentent progressivement en données diverses et variées.
À titre d'exemple, le gouvernement belge dispose de son site de données ouvertes\footnote{\url{http://data.gov.be/}}. Il est actuellement en test et alimenté par les données des services publics fédéraux et flamands. Un site similaire existe pour l'Union Européenne\footnote{\url{http://open-data.europa.eu/}}

Le site WikiLeaks a probablement aidé le journalisme de données à se développer.
Conçu pour héberger des documents secrets qui ont \og fuité \fg, le site
s'est fait connaître en publiant des milliers de documents diplomatiques américains.
Plusieurs médias internationaux (dont \textit{The New York Times, The Guardian, Der Spiegel, El País\dots}) se sont emparés de cette immense masse d'informations, l'ont
retriée, et en ont extrait des informations. Certaines bavures de l'armée américaines
en Irak ont par exemple été révélées au grand public, suite au travail des 
journalistes qui ont exploité les données de WikiLeaks.


\section{Le cycle du journalisme de données}
Comme on l'a ébauché précedemment, les données peuvent intervenir à différentes
étapes du travail journalistique. Cette section détaille ces différentes étapes.

\subsection{Recherche des données}
L'obtention des données peut se faire de différentes manières :
\begin{itemize}
\item Les données peuvent être disponibles sur les portails de données ouvertes évoqués 
plus haut. Elles y sont souvent présentées dans un format exploitable par l'ordinateur. 
On y trouve généralement des données officielles (ce qui ne garantit pas leur 
complétude ou leur exactitude). Si les données recherchées ne sont pas en ligne mais 
que l'on soupçonne leur existence, il est parfois possible de déposer une demande de \og libération des données \fg auprès de l'administration concernée.
\item Les options avancées des moteurs de recherche permettent de cibler un type de 
document recherché (par exemple: les tableaux Excel) ou de se limiter à un site précis. 
Cette techique permet de faire \og remonter\fg des documents noyés dans la masse du web.
\item Il existe aussi de nombreuses bases de données disponibles en ligne (par exemple : DataHub\footnote{\url{http://datahub.io/}} permet de trouver des collections de données, en fonction de mots-clés).
\item Le \textit{scraping} est une technique consistant à récupérer le contenu d'une page web ou d'un ensemble de pages, et d'en extraire les éléments intéressants à 
l'aide d'un programme informatique.
\item Des bases de données non publiques existent : mises en places en interne pour des recherches universitaires, par des passionnés, par des associations... elles nécessitent alors un contact pour y avoir accès.
\item Le \em crowdsourcing\em\, (en français : externalisation des connaissances), repose sur le principe de \og l'appel à la foule \fg et profite de l'aide du public pour
construire ou compléter la base de données.
\end{itemize}

\subsection{Nettoyage des données}

Cette étape technique est facultative, si le journaliste a eu la chance 
de recevoir une collection de données directement exploitable.

Par contre, si les données sont dans un format peu exploitable (exemples : les
documents PDF ont été conçus pour l'impression mais ne sont pas utilisables
directement), il faut parvenir à les encoder dans un tableur ou un logiciel 
de base de données. Dans les cas les plus compliqués, l'assistance d'un 
développeur chevronné sera indispensable.

Un outil comme Open Refine\footnote{\url{http://openrefine.org/}} permet de
supprimer les doublons, de vérifier d'éventuelles fautes de frappe, de regrouper
des données dont le sens est similaire mais dont la représentation informatique
est différente (par exemple : \og Germany \fg et \og Allemagne \fg indiquent le même 
pays, mais il faut le faire comprendre à l'ordinateur).

\subsection{Exploitation des données}

Pour trouver une information pertinente dans un lot de données, il faut d'abord établir
une liste de questions auxquelles on cherche une réponse. Ces questions vont permettre d'angler la recherche. Par exemple, si l'on dispose des chiffres de la Politique Agricole Commune de l'Union européenne, il faut s'avoir si l'on s'intéresse à un ou à plusieurs états, sur quelle période, à toutes les exploitations ou à une catégorie particulière\dots pour pouvoir répondre à la question "Les très petites fermes sont-elles menacées par la PAC?".

Tirer des informations signifiantes de données chiffrées n'est pas une tâche 
évidente. Le journaliste aura d'ailleurs 'tout intérêt à réviser les concepts statistiques de base. 
Nicolas Kayser-Bril rappelle dans \cite{handbookfr} les trois questions à se poser 
devant une série de données:

\begin{enumerate}
\item Comment les données ont-elles été recueillies ? Les données ont-elles été 
fabriquées ou maquillées ? 
\item Que nous apprennent les données ? Par exemple, lorsqu'une moyenne est citée, a-t-
elle du sens ? A-t-on une idée de la distribution de la population?
\item Les données sont-elles fiables ? L'échantillon est-il assez grand et 
représentatif ? Ne confond-on pas corrélation et coïncidence ?
\end{enumerate}

En cas de données non-chiffrées (par exemple : une base de données contenant un carnet
de commandes), il faut parvenir à trier les informations qui peuvent répondre à la 
question posée, réaliser des liens entre différents éléments factuels.
Par exemple, relier un courriel d'une banque des Caraïbes à une société cliente avec 
la liste des administrateurs de la société pour faire apparaître un soupçon de fraude 
fiscale n'est pas une tâche simple, même avec l'aide d'un ordinateur.
Si la collection de données est trop grande, le flair du journaliste pourra l'aider
à cibler un sous-ensemble des données.


L'exploitation des données peut aussi se faire à l'aide du public. Les 400000 notes 
de frais des 
parlementaires britanniques analysées par le \textit{Telegraph} n'auraient pu être
entièrement révisées par la rédaction. Ce sont donc les internautes qui pouvaient
signaler sur le site internet quelles données méritaient un examen plus approfondi. 


\subsection{Communication des données}

Les données peuvent bien sûr servir de base à la rédaction d'un ou plusieurs articles.
Si les données sont numériques, elles peuvent être représentées sous la forme de 
graphiques.

Néanmoins, le fait de disposer de données sous forme informatique permet aussi de 
les intégrer dans des dispositifs de visualisation multimédia.
Les cartes se prêtent bien à cet exerice. Un journaliste qui dispose de la qualité
de l'eau à chaque endroit de baignade en Wallonie peut réaliser une carte interactive
des plages où l'eau est propre. N'oublions pas non plus Adrian Holovaty, dont les \og 
cartes du crime \fg indiquaient précisément les lieux à risque.

Cette visualisation peut aussi faire apparaître une réalité qui ne transparaîssait 
pas directement dans les tablaux de chiffres. 

Notons que bon nombre d'outils simples d'utilisation et souvent gratuits sont 
disponibles sur le web, réduisant ainsi les coûts de développement d'une solution
propre.

Outre les nouvelles possibilités de visualisation, les outils de journalisme de
données permettent aussi d'ajouter l'interactivité. L'internaute ne vient plus
simplement lire un article, ou regarder une carte, destiné à tous. Il peut lui-même
interroger les données. De nombreux sites de média ont ainsi lancé des 
applications informatives qui tentent de répondre directement aux questions des internautes.

On le voit, le journalisme de données pousse à repenser la manière dont l'information 
est communiquée au public. Une des clés, selon Jer Thorp cité dans \cite{handbookfr} 
serait de parvenir à rendre \og humaine \fg une série de chiffres. Le public doit 
percevoir que les informations qu'on lui présente concernent des humains, et ne sont 
pas uniquement des statistiques abstraites. La visualisation et les applications 
interactives permettent de mettre en évidence cette humanité. 

La plupart des spécialistes du journalisme de données recommandent de republier les
données originales -- lorsque cela est possible en regard des lois, notamment celles 
consacrées à la protection de la vie privée, ou en fonction d'une éventuelle licence
d'utilisation sur les données d'origine. Cette republication renforce la transparence
du travail journalistique (chacun peut vérifier les affirmations du journaliste).
Cela participe aussi de l'entraide journalistique : si un confrère cherche des données
sur le sujet concerné, son travail s'en trouvera raccourci.

