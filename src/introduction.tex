% mainfile: document.tex

\chapter*{Introduction}
\addcontentsline{toc}{chapter}{Introduction}
\chaptermark{Introduction}

Exploiter les possibilités des ordinateurs et des bases de données, les compétences des 
développeurs informatiques,
et la souplesse du web pour informer. Voilà qui ne pouvait manquer d'attirer mon 
attention, de par ma double formation d'informaticien et de journaliste. Le journalisme 
de données semble en effet promis à un avenir prometteur, au vu de l'explosion de données 
disponibles sur tous les sujets, et de la démocratisation des outils qui permettent de 
les exploiter et les présenter convivialement.

Le premier chapitre de ce travail tentera d'abord de tracer les contours du  \og journalisme de données \fg, de définir quels sont ses buts et ses méthodes. Ensuite, le deuxième chapitre se focalisera sur son introduction naissante dans les médias belges francophones, et détaillera quelques cas significatifs. Enfin, le troisième chapitre reprend quelques réflexions personnelles sur l'utilisation du journalisme de données.

\vfill

\begin{flushright}
\em
R.M.
\\ Louvain-la-Neuve, septembre 2013. 
\em
\end{flushright}

\vfill

\small
N.B.: Ce travail cite de nombreuses réalisations publiées sur Internet. La version 
électronique de ce document contient des liens cliquables, qui permettront au lecteur 
de visiter plus facilement les différents sites cités.
\normalsize
