% mainfile: document.tex

\chapter*{Conclusion}
\addcontentsline{toc}{chapter}{Conclusion}
\chaptermark{Conclusion}

Le journalisme de données est une pratique encore jeune, qui se base sur des technologies en perpétuelle évolution. Tout en s'inscrivant dans une longue tradition de journalisme d'investigation, il propose de nouveaux outils pour rechercher et traiter l'information. 

Le journalisme de données a déjà montré son utilité démocratique, révélant l'un des plus grands scandales fiscaux des dernières décennies. Ceci n'aurait probablement pas été réalisable dans les mêmes conditions sans les technologies informatiques récentes. Il a aussi pour mérite d'explorer de nouvelles voies d'informer, notamment visuelles et interactives (après tout, ne dit-on pas qu'un bon dessin vaut mieux qu'un long discours?).

Nous avons présenté quelques exemples de son introduction dans les rédactions belges : 
il est certain que ce travail est loin d'être exhaustif. Une rencontre des différentes rédactions, intégrant une analyse plus systématique de leurs productions, de leurs attentes et de leur vision du journalisme de données serait très certainement intéressante, mais sortait du cadre de ce mémoire de master 60. 
