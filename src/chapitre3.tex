% mainfile: document.tex

\chapter{Réflexion critique}

Le journalisme de données ouvre des portes intéressantes et innovantes en termes de
recherche et de présentation de l'information, comme décrit dans les chapitres 
précédents. Ce chapitre contient quelques réflexions personnelles à propos du journalisme 
de données. Ces questions ne sont probablement pas toutes élucidables immédiatement
au vu de la jeunesse du journalisme de données.

Les auteurs spécialisés l'ont souligné \cite{handbook, bradshaw, meyer}, travailler
avec des données demande une certaine formation à la compréhension des chiffres. Un 
journaliste qui ne perçoit pas la différence entre une moyenne et une médiane, 
ou entre une corrélation et une coïncidence risque de se diriger vers des conclusions
fausses. Des compétences de base en programmation informatiques permettent aussi
une meilleure compréhension entre un journaliste et un développeur.

De plus, étant donné que la population de journalistes ne compte pas que des 
\textit{aficionados} 
des nombres, si une rédaction encourage le journalisme de données, cela ne risque-t-il 
pas de créer une fracture dans la rédaction entre les tenants du journalisme de données 
et les autres? Ou bien de voir se créer des rédactions \og pro \fg et \og anti \fg 
journalisme de données? Comme les développeurs informatiques sont-ils perçus par 
les journalistes (concurrents ou collègues) ?

Dans le même sens, on peut se demander quelles sont les rédactions qui oseront 
engager des moyens financiers importants dans du travail journalistique basé sur les 
données ? L'exemple de \textit{L'Avenir} montre que le journalisme de données est 
pratiquable à petite échelle, mais l'on peut douter que des projets de grande ampleur,
impliquant d'importants développements informatiques, aient un coût aussi modéré. Ceci 
dit, cette critique peut s'adresser à toute forme d'enquête journalistique longue,
qu'elle soit réalisée avec des données ou non.

Du point de vue des \og produits \fg, la qualité de la mise en forme
(visualisation, \textit{mashup}, application interactive\dots) me semble un des enjeux
fondamentaux du journalisme de données. Un long projet d'exploration de données
risque de passer inaperçu s'il est mal présenté ou difficile à partager sur
les réseaux sociaux. 

Enfin, le journalisme de données ouvre de nouvelles pistes pour raconter l'information.
Outre la non-linéarité de la plupart des \textit{mashups}, les applications interactives
permettent au public de poser lui-même des questions, d'interroger lui-même les données (dans les limites des capacités de l'application). Le public peut alors devenir un peu plus \og acteur de l'information \fg.

